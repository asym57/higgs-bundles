\documentclass[12pt,letterpaper,reqno]{amsart}

% \usepackage{mathtools}
\usepackage{epsfig}
\usepackage{amsmath}
\usepackage{amssymb}
\usepackage{amsthm}
\usepackage{indentfirst}
\usepackage{xspace}
\usepackage{multirow}
\usepackage{hyperref}
\usepackage{xcolor}
\usepackage{verbatim}
\usepackage[letterpaper,margin=1in,headheight=15pt]{geometry}
\usepackage{mathpazo}
\usepackage{tikz-cd}
\usepackage{booktabs}
\usepackage{framed}
\usepackage{float}
\usepackage{thmtools}
\usepackage{dashrule}
\usepackage[missing=]{gitinfo2}

\definecolor{darkred}{rgb}{0.5,0.1,0.1}
\definecolor{darkgreen}{rgb}{0,0.55,0.2}
\hypersetup{colorlinks=true,urlcolor=darkred,linkcolor=darkred,citecolor=darkred}
\definecolor{shadecolor}{rgb}{0.85,0.85,0.85}

% Bibliography formatting
\usepackage[bibstyle=authoryear-comp,labeldate=false,defernumbers=true,maxnames=20,firstinits=true,uniquename=init,dashed=false,backend=biber,sorting=none]{biblatex}

\DeclareNameAlias{sortname}{first-last}

\DeclareFieldFormat{url}{\url{#1}}
\DeclareFieldFormat[article]{pages}{#1}
\DeclareFieldFormat[inproceedings]{pages}{\lowercase{pp.}#1}
\DeclareFieldFormat[incollection]{pages}{\lowercase{pp.}#1}
\DeclareFieldFormat[article]{volume}{\textbf{#1}}
\DeclareFieldFormat[article]{number}{(#1)}
\DeclareFieldFormat[article]{title}{\MakeCapital{#1}}
\DeclareFieldFormat[inproceedings]{title}{#1}
\DeclareFieldFormat{shorthandwidth}{#1}

% Don't use "In:" in bibliography. Omit urls from journal articles.
\DeclareBibliographyDriver{article}{%
  \usebibmacro{bibindex}%
  \usebibmacro{begentry}%
  \usebibmacro{author/editor}%
  \setunit{\labelnamepunct}\newblock
  \MakeSentenceCase{\usebibmacro{title}}%
  \newunit
  \printlist{language}%
  \newunit\newblock
  \usebibmacro{byauthor}%
  \newunit\newblock
  \usebibmacro{byeditor+others}%
  \newunit\newblock
  \printfield{version}%
  \newunit\newblock
%  \usebibmacro{in:}%
  \usebibmacro{journal+issuetitle}%
  \newunit\newblock
  \printfield{note}%
  \setunit{\bibpagespunct}%
  \printfield{pages}
  \newunit\newblock
  \usebibmacro{eprint}
  \newunit\newblock
  \printfield{addendum}%
  \newunit\newblock
  \usebibmacro{pageref}%
  \usebibmacro{finentry}}

% Remove dot between volume and number in journal articles.
\renewbibmacro*{journal+issuetitle}{%
  \usebibmacro{journal}%
  \setunit*{\addspace}%
  \iffieldundef{series}
    {}
    {\newunit
     \printfield{series}%
     \setunit{\addspace}}%
  \printfield{volume}%
%  \setunit*{\adddot}%
  \printfield{number}%
  \setunit{\addcomma\space}%
  \printfield{eid}%
  \setunit{\addspace}%
  \usebibmacro{issue+date}%
  \newunit\newblock
  \usebibmacro{issue}%
  \newunit}


% Bibliography categories
\def\makebibcategory#1#2{\DeclareBibliographyCategory{#1}\defbibheading{#1}{\section*{#2}}}
\makebibcategory{books}{Books}
\makebibcategory{papers}{Refereed research papers}
\makebibcategory{chapters}{Book chapters}
\makebibcategory{conferences}{Papers in conference proceedings}
\makebibcategory{techreports}{Unpublished working papers}
\makebibcategory{bookreviews}{Book reviews}
\makebibcategory{editorials}{Editorials}
\makebibcategory{phd}{PhD thesis}
\makebibcategory{subpapers}{Submitted papers}
\makebibcategory{curpapers}{Current projects}

\setlength{\bibitemsep}{2.65pt}
\setlength{\bibhang}{.8cm}
\renewcommand{\bibfont}{\small}

\renewcommand*{\bibitem}{\addtocounter{papers}{1}\item \mbox{}\hskip-0.85cm\hbox to 0.85cm{\hfill\arabic{papers}.~~}}
\defbibenvironment{bibliography}
{\list{}
  {\setlength{\leftmargin}{\bibhang}%
   \setlength{\itemsep}{\bibitemsep}%
   \setlength{\parsep}{\bibparsep}}}
{\endlist}
{\bibitem}

\newenvironment{publications}{\section{\LARGE Publications}\label{papersstart}\vspace*{0.2cm}\small
\titlespacing{\section}{0pt}{1.5ex}{1ex}\itemsep=0.00cm
}{\label{papersend}\addtocounter{sumpapers}{-1}\refstepcounter{sumpapers}\label{sumpapers}}

\def\printbib#1{\printbibliography[category=#1,heading=#1]\lastref{sumpapers}}

% Counters for keeping track of papers
\newcounter{papers}\setcounter{papers}{0}
\newcounter{sumpapers}\setcounter{sumpapers}{0}
\def\lastref#1{\addtocounter{#1}{\value{papers}}\setcounter{papers}{0}}

% theorem environments
\declaretheoremstyle[spaceabove=0.25cm,spacebelow=0.25cm,notefont=\normalfont\bfseries, notebraces={(}{)}]{theorem}
\declaretheoremstyle[spaceabove=0.25cm,spacebelow=0.25cm,bodyfont=\normalfont,notefont=\normalfont\bfseries, notebraces={(}{)}]{noital}
\declaretheoremstyle[spaceabove=0.25cm,spacebelow=0.25cm,bodyfont=\normalfont\color{darkgreen},notefont=\normalfont\bfseries, notebraces={(}{)}]{green}

\declaretheorem[name=Theorem,numberwithin=section,style=theorem]{thm}
\declaretheorem[name=Proposition,sibling=thm,style=theorem]{prop}
\declaretheorem[name=Corollary,sibling=thm,style=theorem]{cor}
\declaretheorem[name=Lemma,sibling=thm,style=theorem]{lem}
\declaretheorem[name=Definition,sibling=thm,style=noital]{defn}
\declaretheorem[name=Example,sibling=thm,style=noital]{example}
\declaretheorem[name=Exercise,numberwithin=section,style=green]{exercise}
\declaretheorem[name=Proof,style=noital,numbered=no]{pf}

\numberwithin{equation}{section}


% macros for convenience
\newcommand{\tops}{\texorpdfstring}

\newcommand{\nid}{\noindent}

\newcommand{\fa}{{\mathfrak a}}
\newcommand{\fp}{{\mathfrak p}}
\newcommand{\fk}{{\mathfrak k}}
\newcommand{\fg}{{\mathfrak g}}
\newcommand{\fh}{{\mathfrak h}}
\newcommand{\fn}{{\mathfrak n}}
\newcommand{\fq}{{\mathfrak q}}
\newcommand{\fm}{{\mathfrak m}}
\newcommand{\fr}{{\mathfrak r}}

\newcommand{\cC}{\ensuremath{\mathcal C}}
\newcommand{\cG}{\ensuremath{\mathcal G}}
\newcommand{\cB}{\ensuremath{\mathcal B}}
\newcommand{\cL}{\ensuremath{\mathcal L}}
\newcommand{\cS}{\ensuremath{\mathcal S}}
\newcommand{\cF}{\ensuremath{\mathcal F}}
\newcommand{\cK}{\ensuremath{\mathcal K}}
\newcommand{\cZ}{\ensuremath{\mathcal Z}}
\newcommand{\cM}{\ensuremath{\mathcal M}}
\newcommand{\cO}{\ensuremath{\mathcal O}}
\newcommand{\cH}{\ensuremath{\mathcal H}}
\newcommand{\cX}{\ensuremath{\mathcal X}}
\newcommand{\cY}{\ensuremath{\mathcal Y}}
\newcommand{\cA}{\ensuremath{\mathcal A}}
\newcommand{\cI}{\ensuremath{\mathcal I}}

\newcommand{\R}{\ensuremath{\mathbb R}}
\newcommand{\C}{\ensuremath{\mathbb C}}
\newcommand{\PP}{\ensuremath{\mathbb P}}
\newcommand{\Z}{\ensuremath{\mathbb Z}}
\newcommand{\Q}{\ensuremath{\mathbb Q}}
\newcommand{\A}{\ensuremath{\mathbb A}}
\newcommand{\bbH}{\ensuremath{\mathbb H}}
\newcommand{\bbI}{\ensuremath{\mathbb I}}
\newcommand{\bS}{\ensuremath{\mathbb S}}

\newcommand{\half}{\ensuremath{\frac{1}{2}}}
\newcommand{\qtr}{\ensuremath{\frac{1}{4}}}
\newcommand{\bq}{{\mathbf q}}
\newcommand{\N}{{\mathcal N}}
\newcommand{\F}{{\mathcal F}}
\newcommand{\HH}{{\mathcal H}}
\newcommand{\LL}{{\mathcal L}}
\newcommand{\RR}{{\mathcal R}}
\newcommand{\V}{{\mathcal V}}
\newcommand{\dirac}{\!\!\not\!\partial}
\newcommand{\Dirac}{\!\!\not\!\!D}
\newcommand{\cE}{{\mathcal E}}
\newcommand{\vs}{\not\!v}
\newcommand{\kahler}{K\"ahler\xspace}
\newcommand{\hk}{hyperk\"ahler\xspace}
\newcommand{\Hk}{Hyperk\"ahler\xspace}
\newcommand{\del}{\ensuremath{\partial}}
\newcommand{\delbar}{\ensuremath{\overline{\partial}}}
\newcommand{\I}{{\mathrm i}}
\newcommand{\J}{{\mathrm j}}
\newcommand{\K}{{\mathrm k}}
\newcommand{\e}{{\mathrm e}}
\newcommand{\de}{\mathrm{d}}
\newcommand{\ab}{\mathrm{ab}}
\newcommand{\vol}{\mathrm{vol}}

\newcommand{\abs}[1]{\lvert#1\rvert}
\newcommand{\norm}[1]{\lVert#1\rVert}
\newcommand{\IP}[1]{\langle#1\rangle}
\newcommand{\dwrt}[1]{\frac{\partial}{\partial#1}}
\newcommand{\eps}{\epsilon}

\newcommand{\ti}[1]{\textit{#1}}

\DeclareMathOperator{\ad}{ad}
\DeclareMathOperator{\im}{Im}
\DeclareMathOperator{\re}{Re}
\DeclareMathOperator{\Tr}{Tr}
\DeclareMathOperator{\End}{End}
\DeclareMathOperator{\Hom}{Hom}
\DeclareMathOperator{\Aut}{Aut}
\DeclareMathOperator{\Sym}{Sym}
\DeclareMathOperator{\diag}{diag}
\DeclareMathOperator{\Bun}{Bun}

\newcommand{\insfig}[2]{\begin{figure}[htbp] \centering \includegraphics[scale=#2]{figures/#1-crop.pdf} \label{fig:#1} \end{figure}}
% syntax: \insfig{name}{0.5}{caption}

\newcommand{\fixme}[1]{{\color{blue}{[#1]}}}
\newcommand{\currentposition}{{\color{blue} \noindent\makebox[\linewidth]{\hdashrule{\paperwidth}{1pt}{3mm}}}}

% \mathtoolsset{showonlyrefs}

\bibliography{higgs-bundles}

\begin{document}

\setcounter{page}{1}

{\noindent \tiny \color{gray} \tt \gitAuthorIsoDate \hfill
\gitAbbrevHash}

\begin{center} 
{\textbf {{Moduli of Higgs Bundles}}} \\
Preliminary draft
\end{center}

\section{Introductory motivation}

Suppose given a compact Riemann surface $C$
of genus $g \ge 2$ and a compact reductive
Lie group $G$, e.g. $G = U(1)$, $G = SU(2)$. 
Built from these data there is a
moduli space $$\cM = \cM^H(C,G)$$
It is \ti{almost} a manifold --- has some singularities, but also
some components without singularities, and at first we can focus
on the parts without singularities.
It can be seen in various ways:

\begin{itemize}
\item $\cM$ is the (twisted) \ti{character variety}, i.e. moduli space of
(twisted) 
reductive representations\footnote{``Reductive'' means the closure of 
the image is a reductive subgroup of $G_\C$.}
$\pi_1(C) \to G_\C$.
e.g. for $g=2$ and $G = SU(2)$, 
this means
\begin{equation}
  \cM = \{ A_1,A_2,B_1,B_2 \in SL(2,\C): A_1 B_1 A_1^{-1} B_1^{-1} A_2 B_2 A_2^{-1} B_2^{-1} = \pm 1 \} / \sim
\end{equation}
\item $\cM$ is the moduli space parameterizing (stable) \ti{flat $G_\C$-connections}
over $C$. (Certain sheaves on this moduli space are 
basic objects on ``$B$ side'' of the geometric Langlands correspondence.)
\item $\cM$ is a partial compactification of $T^* \Bun(C,G)$, where $\Bun(C,G)$ is the moduli space of semistable $G$-bundles on $C$. (Lagrangian submanifolds are related to \ti{D-modules} on $\Bun(C,G)$, basic objects on ``$A$ side'' 
of the geometric Langlands correspondence.)
\insfig{higgs-bundles-2}{0.55}
\item $\cM$ is a \ti{complex integrable system} \cite{MR88i:58068}, i.e. a holomorphic
symplectic space fibered over a complex base with Lagrangian
fibers, generic fiber a compact complex torus. \insfig{higgs-bundles-1}{0.8}
\item $\cM$ is a noncompact \ti{Calabi-Yau space}, i.e. a \kahler
space admitting a Ricci-flat metric, in some sense a close 
cousin of the K3 surface; from this point of view
it is a paradigmatic example of the Strominger-Yau-Zaslow
philosophy \cite{Strominger:1996it}, 
which says that every Calabi-Yau space arises naturally
as a special Lagrangian \ti{torus fibration} over a complex base,
and that its \ti{mirror} can be obtained by a natural fiberwise duality
operation; moreover in this case the mirror is a space of the same kind,
namely $\cM^\vee = \cM^H(C,^L G)$ where 
$^L G$ is the \ti{Langlands dual} group \cite{mlh,MR2957305}.
(The mirror symmetry
exchanges the two sides of the geometric Langlands correspondence.)
\item $\cM$ is a \ti{cluster variety}, built by gluing together
very simple pieces $(\C^\times)^n$ in an essentially \ti{combinatorial} way. (Almost: to make this precisely true, we have to include \ti{punctures} on $C$; but even without the punctures, some cluster-like
structure seems to persist.)
\item $\cM$ is the space of solutions of an interesting PDE, \ti{Hitchin's equations} \cite{MR89a:32021}, containing as special cases various sorts of harmonic maps (including \ti{uniformization} in the case $G = PSU(2)$).
\end{itemize}

How can one space $\cM$ be so many different things at once?

A partial answer comes from another structure $\cM$ carries, namely
the \ti{\hk} structure. This says in short that $\cM$ has a 
metric compatible with many
different complex structures,
fitting together in a specific way; thus $\cM$ gives rise to
many complex manifolds which look quite different from one
another, but are nevertheless canonically diffeomorphic.
Loosely speaking, one complex structure comes from the Riemann 
surface $C$, another comes from $G_\C$.
An \hk structure is rather rigid and gives a lot of constraints, e.g.
it implies that the metric on $\cM$ is Ricci-flat, and even lets 
us say some things about what the metric looks like (much
more than we can say for ``generic'' Ricci-flat metrics);
also allows us to study the \ti{topology} of $\cM$, e.g. 
its Betti numbers.

Our first major aim is to understand this structure --- first 
we will study some simpler ``baby'' examples of \hk geometry, then
we will study $\cM(C,G)$ for $G = U(1)$, finally we will
come to $\cM(C,G)$ for general $G$.

(A fuller answer should come from the way $\cM$
fits into supersymmetric quantum field theory; but this is 
mostly beyond the scope of this course.)



\section{Local complex and \kahler geometry: a quick review}

This is only intended as a review and to fix notation.
There are many references for this material: one
good one is \cite{MR2093043}.


\subsection{Complex manifolds} 
In this section $X$ is a smooth manifold.

\begin{defn}[Almost complex structure]
An \ti{almost complex structure} on $X$ is a smooth section $I$ of $\End(TX)$ with $I^2 = -1$. An \ti{almost complex manifold} is a pair $(X,I)$ where $I$ is an almost complex structure.
If $X$ has real dimension $2n$, an almost complex
structure $I$ equips $TX$ with the structure of
\ti{complex} vector bundle over $X$, of rank $n$,
and we say the \ti{complex dimension} $\dim_\C X$ is
$n$.
\end{defn}

\begin{example}[Flat complex space] $\C^n$ has a canonical almost complex structure $I$, as follows. Each tangent space $T_p \C^n \simeq \C^n$ canonically; $I$ is multiplication by $\I$, thought of as an 
endomorphism of the underlying $2n$-dimensional real vector space.
Writing $z_i = x_i + \I y_i$, and taking the coordinate basis
$\{\partial_{x_1}, \partial_{x_2}, \dots, \partial_{x_n}, \partial_{y_1}, \partial_{y_2}, \dots, \partial_{y_n}\}$ for for $T_p \C^n$, $I$ is represented by the matrix
\begin{equation}
   I = \begin{pmatrix} {\bf 0}_{n \times n} & -{\bf 1}_{n \times n} \\ {\bf 1}_{n \times n} & {\bf 0}_{n \times n} \end{pmatrix}.
\end{equation}
\end{example}

\begin{defn}[Holomorphic maps] \label{def:holomorphic-maps} If $(X,I_X)$ and $(Y,I_Y)$ are almost
complex manifolds, a \ti{holomorphic map} $\phi: X \to Y$
is one obeying
\begin{equation}
 I_Y \circ \de \phi = \de \phi \circ I_X.  
\end{equation}
\end{defn}

\begin{exercise} Show that, if both $(X,I_X)$ and $(Y,I_Y)$ are $\C$ with its canonical 
almost complex structure, \autoref{def:holomorphic-maps} becomes the standard
definition of holomorphic function (Cauchy-Riemann equations).
\end{exercise}

\begin{defn}[Complex structures] \label{def:complex-structures} An almost complex structure $I$ on $X$ is \ti{integrable}, 
or a \ti{complex structure}, if there is a covering of $X$ by open sets $U_\alpha$ with holomorphic diffeomorphisms $\phi_\alpha: U_\alpha \to V_\alpha \subset \C^n$ (where on $\C^n$ we take the
canonical almost complex structure.) A \ti{complex manifold} is an almost complex manifold $(X,I)$ with $I$ integrable.
\end{defn}

\begin{exercise} Show that \autoref{def:complex-structures} is equivalent to the usual definition of a complex manifold as a space
$X$ with a covering by charts $\phi_\alpha: U_\alpha \to \C^n$, where the transition maps are holomorphic (obey Cauchy-Riemann equations).
\end{exercise}

\begin{example}[Complex structure on $\C^n$] A tautological example is $X = \C^n$ itself with its 
canonical almost complex structure: just take a single open set $U = \C^n$, and $\phi: U \to \C^n$ to be the identity map. So the canonical almost complex structure on $\C^n$ is, tautologically, a
complex structure.
\end{example}

There are various equivalent ways of formulating the integrability condition. One which will be useful for us is:
\begin{prop}[Integrability means vanishing of Nijenhuis tensor] Define the \ti{Nijenhuis tensor} $N_I \in \Omega^0(\wedge^2 T^*X \otimes TX)$ as a map
$T^*X \otimes T^*X \to TX$ by
\begin{equation}
N_I(v,w) = [\tilde v,\tilde w] + I[\tilde v,I \tilde w] + I[I \tilde v, \tilde w] - [I\tilde v,I\tilde w],
\end{equation}
where $\tilde v$ and $\tilde w$ are any vector fields extending $v$, $w$.
Then $I$ is integrable if and only if
\begin{equation}
N_I = 0.  
\end{equation}
\end{prop}
\begin{pf} To show that integrability implies $N_I = 0$ is straightforward
by directly computing in a local coordinate chart. The other direction
is much harder --- it is the content of the Newlander-Nirenberg theorem.
\end{pf}

\subsection{Type decompositions}
Suppose $(X,I)$ is an almost complex manifold.
We have a decomposition
of $T_\C X = TX \otimes_\R \C$,
\begin{equation}
T_\C X = T^{1,0} X \oplus T^{0,1} X
  \end{equation}
where $T^{1,0} X$ and $T^{0,1} X$ are 
respectively the
$+\I$ and $-\I$ eigenspaces of $I$.
Both $TX$ and $T^{1,0}X$ are complex vector bundles 
of rank $n$; it is sometimes convenient to identify
them, by projection on the $(1,0)$ part.

\begin{exercise}
Show that this projection $\pi: TX \to T^{1,0}X$ indeed is an isomorphism 
of complex vector bundles. (This reduces essentially
to a question of linear algebra, concerning 
a vector space $V$ with complex structure $I$,
and its complexification $V_\C$.) If $X = \C$, 
what are $\pi(\partial_x)$ and $\pi(\partial_y)$?
\end{exercise}

There is also a dual decomposition
\begin{equation}
T^*_\C X = (T^*)^{1,0} X \oplus (T^*)^{0,1} X,
  \end{equation}
where $(T^*)^{1,0} X$ is the annihilator of
$T^{0,1} X$, and $(T^*)^{0,1} X$ is the annihilator of
$T^{1,0} X$.
This decomposition induces
\begin{equation}
\wedge^* T^*_\C X = \bigoplus_{p,q=0}^n \wedge^{p,q} T^* X , \qquad \Omega^*_\C X = \bigoplus_{p,q=0}^n \Omega^{p,q}(X).
\end{equation}

\begin{prop}[Integrability versus type decompositions]
Suppose $(X,I)$ is an almost complex manifold. The following are equivalent:
\begin{itemize}
\item $I$ is integrable.
\item There is a decomposition
\begin{equation}
\de = \partial + \bar\partial, \quad \partial: \Omega^{p,q}(X) \to \Omega^{p+1,q}(X), \quad \bar\partial: \Omega^{p,q}(X) \to \Omega^{p,q+1}(X).
\end{equation}
\item The distribution $T^{0,1}X$ is integrable: if $v$, $w$ are 
sections of $T^{0,1}X$ then $[v,w]$ is also a section of $T^{0,1}X$.
\end{itemize}
\end{prop}

Complex conjugation is an $\R$-linear map
$\Omega^{p,q}(X) \to \Omega^{q,p}(X)$;
thus it maps $\Omega^{p,p}(X)$ to itself;
we let $\Omega_\R^{p,p}(X)$ denote the fixed subspace.

\subsection{Holomorphic vector bundles}

In this section $(X,I)$ is always a complex manifold.

\begin{defn}[Holomorphic vector bundle] A \ti{holomorphic vector bundle} over $X$ is a complex vector bundle $E$ over $X$, equipped with an operator
\begin{equation}
  \bar\partial_E :  \Omega^{p,q}(E) \to \Omega^{p,q+1}(E)
\end{equation}
obeying
\begin{equation}
  \bar\partial_E (\alpha \psi) = (\bar\partial \alpha) \psi + (-1)^{\abs{\alpha}} \alpha \wedge \bar\partial_E \psi \qquad \alpha \in \Omega^*(X), \quad \psi \in \Omega^0(E)
\end{equation}
and the integrability condition
\begin{equation}
  \bar\partial_E^2 = 0.
\end{equation}
\end{defn}

It is useful to think of $\bar\partial_E$ as a kind of 
partially-defined flat connection, which allows us to differentiate
only in the $(0,1)$ ``directions.''
The structure of holomorphic vector bundle is much 
more rigid than that of a merely complex vector bundle.
We emphasize that this structure makes sense only when $X$ is 
a complex manifold, while complex vector bundles make sense 
over any $X$.

\begin{prop}[Equivalence of definitions of holomorphic vector bundle] A structure of holomorphic vector bundle on $E$ is equivalent to a maximal atlas 
of preferred trivializations of $E$, such that
the transition maps 
$U_\alpha \cap U_\beta \to GL(r,\C)$ 
are holomorphic. 
\end{prop}
\begin{pf}
This is a sort of linear analogue of the 
Newlander-Nirenberg theorem; for a proof see 
\cite{MR1079726} Theorem 2.1.53, proven in Section 2.2.
\end{pf}

% \begin{example} The trivial bundle $X \times \C^r$
% carries a canonical holomorphic structure.
% \end{example}

\begin{example}[Tangent bundle as a holomorphic bundle] The tangent bundle $TX$ carries
a canonical structure of holomorphic vector bundle.
Indeed, the holomorphic charts $\phi_\alpha = (z_1, \dots, z_n)$ 
give rise to preferred trivializations corresponding 
to the bases $\{\partial_{z_1}, \dots, \partial_{z_n} \}$
for $T^{1,0} X \simeq TX$, and the transition 
maps are given by the Jacobian matrices 
$(\partial z'_i / \partial z_j)_{i,j=1}^n$, which are holomorphic.
% It can be described as follows: for each coordinate
% chart $\phi_\alpha: U \to \C^n$, we get an isomorphism
% $\de \phi_\alpha: TU \to T \C^n$, and under this 
% isomorphism $\bar\partial_{TX}$ is identified
% with the canonical holomorphic structure.
\end{example}

\begin{defn}[Connection compatible with holomorphic structure]
If $E$ is a holomorphic vector bundle over $X$,
a connection $D$ in $E$ is \ti{compatible with the
holomorphic structure} if, for all 
$\psi \in \Omega^0(E)$, the $(0,1)$ part of $D \psi$ 
is $\bar\partial_E \psi$.
\end{defn}

\begin{defn}[Chern connection]
If $E$ is a holomorphic vector bundle over $X$ with
a Hermitian metric $h$, the \ti{Chern connection} 
in $E$ is the unique connection which is 
$h$-unitary and compatible with the holomorphic
structure.
\end{defn}

\subsection{Hermitian and \kahler metrics}
Throughout this section $(X,I)$ is an almost complex manifold.

\begin{defn}[Hermitian metric on complex manifold] A Hermitian metric on 
$(X,I)$ is a Riemannian metric $g$ obeying
$$ g(v,w) = g(Iv,Iw). $$
Equivalently, with respect to the decomposition 
\begin{equation}
  \Sym^2 (T_\C X) = \Sym^{2,0} TX \oplus \Sym^{1,1} TX \oplus \Sym^{0,2} TX,
\end{equation}
we have $g \in \Sym^{1,1} TX$, i.e. $g$ is of ``type $(1,1)$.''
\end{defn}

\begin{defn}[Fundamental form] If $g$ is a Hermitian metric on $(X,I)$, 
the \ti{fundamental form} $\omega \in \Omega^{1,1}_\R(X)$
is
\begin{equation}
  \omega(v,w) = g(Iv,w).
\end{equation}
\end{defn}

\begin{exercise}If $g$ is a Hermitian metric on $(X,I)$ check that
\begin{equation}
  {\mathrm {vol}}_g = \frac{\omega^n}{n!}
\end{equation}
(in particular, $\omega$ is nondegenerate.)
\end{exercise}

The term ``Hermitian'' might seem confusing here since $g$ is just 
an ordinary real-valued metric on the real vector bundle 
$TX$. The following should help:

\begin{exercise} \label{exc:hermitian-metric} 
If $g$ is a Hermitian metric on $(X,I)$,
verify that
\begin{equation}
  h = g - \I \omega
\end{equation}
defines a Hermitian metric
on the complex vector bundle $TX$.
(Our convention is that Hermitian metrics are
$\C$-linear in the \ti{first} slot.)
\end{exercise}

Let $\nabla$ denote the 
Levi-Civita connection on $TX$ induced
by the metric $g$.

\begin{defn}[\kahler metric] If $g$ is a Hermitian metric 
on $(X,I)$, we say $g$ is \ti{\kahler} if
\begin{equation}
  \nabla I = 0.
\end{equation}
Then $(X,g,I)$ is a \ti{\kahler manifold},
and $\omega$ is the \ti{\kahler form}.
\end{defn}

The \kahler property has various useful alternative 
characterizations:
% \begin{prop}[\kahler means covariant constancy of $I$]
% If $g$ is a Hermitian metric on $X$,
% $g$ is \kahler if and only if
% $\nabla I = 0$.
% \end{prop}

\begin{prop}[Characterizations of \kahler metrics] \label{cor:kahler-cc}
If $g$ is a Hermitian metric on $(X,I)$, with 
fundamental form $\omega$, then the following are
equivalent:
\begin{enumerate}
  \item $g$ is \kahler,
  \item $\nabla I = 0$,
  \item $\nabla \omega = 0$,
  \item $I$ is integrable and $\nabla$ agrees with the Chern connection on $TX$,
  when we view $TX$ as a holomorphic vector bundle, with the induced Hermitian 
  metric $h$ of \autoref{exc:hermitian-metric},
  \item $I$ is integrable and $\de \omega = 0$.
\end{enumerate}
\end{prop}

\begin{pf} As we defined \kahler we have 
automatically $(1) \Leftrightarrow (2)$. 
Using $\nabla g = 0$ we easily obtain $(2) \Leftrightarrow (3)$.
So all of $(1)$-$(3)$ are equivalent.

Now we consider $(4)$.
To show $(2)$ implies integrability of $I$,
note that for $v,w$ sections of $T^{0,1} X$ we have
(using the torsion-free property of $\nabla$)
\begin{equation}
  I[v,w] = I(\nabla_v w - \nabla_w v) = \nabla_v(Iw) - \nabla_w(Iv) = -\I (\nabla_v w - \nabla_w v) = -\I [v,w].
\end{equation}
Also $(2)$ implies that
for $v \in TX$ and $w$ a section of $T^{1,0} X$
we have $I (\nabla_v w) = \nabla_v (Iw) = \I \nabla_v w$,
so that $\nabla_v w$ is also a section of $T^{1,0} X$;
this means $\nabla$ is compatible with the holomorphic
structure.
Finally $(2)+(3)$ implies $\nabla h = 0$.
So we have shown that $(2) \Rightarrow (4)$.
Conversely, we have easily $(4) \Rightarrow (3)$,
since $\omega$ is the imaginary part of $h$.
Thus, all of $(1)$-$(4)$ are equivalent.

Finally we consider $(5)$. We already showed that
$(2)$ implies integrability of $I$. Also
$(3)$ immediately implies $\de \omega = 0$.
Thus we have $(2)+(3) \Rightarrow (5)$.
All that remains is to see that $(5) \Rightarrow (4)$,
which is the most interesting part. This amounts to verifying
that the Chern connection is torsion-free (then it will have
to agree with $\nabla$, since $\nabla$ is the unique connection
in $TX$ which is torsion-free and has $\nabla g = 0$.) \fixme{...}
\end{pf}




In particular \autoref{cor:kahler-cc} implies that any complex 
submanifold of a \kahler manifold is again \kahler. Combining this 
with the fact that $\C\PP^n$ admits a \kahler metric
(Fubini-Study), we obtain a huge supply of examples.

Finally we quickly recall the notion of special holonomy. Recall that for any Riemannian metric $g$ the parallel transport of Levi-Civita preserves $g$, so that 
for any $p \in X$ the holonomy group $Hol_\nabla(p) \subset GL(T_p X)$ 
is contained in the subgroup $O(g(p)) \simeq O(2n)$.
For a \kahler metric, \autoref{cor:kahler-cc} says
the parallel transport of Levi-Civita preserves the Hermitian metric
$h$ on the complex vector bundle $TX$. Thus, for any $p \in X$, the holonomy group
$Hol_\nabla(p) \subset GL(T_p X)$ is contained in the smaller group $U(h(p)) \simeq U(n)$.
Conversely, if $Hol_\nabla(p)$ is contained in some subgroup 
isomorphic to $U(n)$ then it preserves some $h$, from 
which one can prove:

\begin{prop}[Special holonomy of \kahler manifolds] Given any Riemannian metric $g$ on a manifold $M$ of dimension $2n$, $g$ is a \kahler metric 
(for some complex structure $I$ on $M$) if and only if
the holonomy group at a point is contained in
a subgroup isomorphic to $U(n)$.
\end{prop}


\section{Hyperk\"ahler manifolds}

Useful (and inspiring) references are \cite{MR88f:53087,Hitchin-hk,MR1798605,boalch-notes}.

\subsection{Basic definitions}

\begin{defn}[\Hk manifold] A \ti{\hk manifold} is 
a tuple $(X,g,I_1,I_2,I_3)$, where $(X,g)$ is a Riemannian 
manifold equipped with three complex 
structures $I_i$ obeying $I_1 I_2 = I_3$,
such that $(X,g,I_i)$ is \kahler for $i=1,2,3$.
\end{defn}

It is crucial that we require the \ti{single} metric
$g$ to be \kahler for \ti{all} of the $I_i$:
this is a very strong condition!
We denote the three corresponding
\kahler forms $\omega_i$.
Sometimes it is convenient to use instead the
notation $(I_1,I_2,I_3) = (I,J,K)$
and $(\omega_1,\omega_2,\omega_3) = (\omega_I,\omega_J,\omega_K)$.

\begin{exercise} Show that the relations $I_1 I_2 = I_3$ and $I_1^2 = I_2^2 = I_3^2 = -1$ are equivalent to the full set of quaternion relations
\begin{gather}
  I_1 I_2 = I_3, \quad I_2 I_1 = - I_3, \\
  I_2 I_3 = I_1, \quad I_3 I_2 = - I_1, \\
  I_3 I_1 = I_2, \quad I_1 I_3 = - I_2, \\
  I_1^2 = I_2^2 = I_3^2 = -1.
\end{gather}
\end{exercise}

\begin{defn}[Holomorphic symplectic form] If $(X,I)$ is a complex manifold, 
$\Omega \in \Omega^{2,0}(X)$ is a \ti{holomorphic
symplectic form} if $\de \Omega = 0$
and $\Omega$ is nondegenerate in the holomorphic sense, i.e. it induces
an isomorphism $T^{1,0}X \to (T^{1,0} X)^*$.
\end{defn}
(Note that this definitely does \ti{not} mean that $\Omega$ is nondegenerate on the whole $T_\C X$. Indeed, since $\Omega$ is 
of type $(2,0)$ its contraction with any $v \in T^{0,1} X$ vanishes.)

\begin{prop}[\Hk manifolds are holomorphic symplectic] If $X$ is \hk then $\Omega_1 = \omega_2 + \I \omega_3$ is a \ti{holomorphic symplectic form} with respect to structure $I_1$ (and similarly with the indices 
$1,2,3$ cyclically permuted.)
\end{prop}
\begin{pf} 
\begin{align}
    \Omega_1(v,w) &= \omega_2(v,w) + \I \omega_3(v,w) \\
    &= g(I_2 v,w) + \I g(I_3 v,w)
\end{align}
Thus
\begin{align}
    \Omega_1(I_1 v,w) &= g(I_2 I_1 v, w) + \I g(I_3 I_1 v,w) \\
    &= -g(I_3 v, w) + \I g(I_2 v, w) \\
    &= \I \Omega_1(v,w)
\end{align}
and similarly
\begin{equation}
 \Omega_1(v,I_1 w) = \I \Omega_1(v,w). 
\end{equation}
It follows that $\Omega_1$ is of type $(2,0)$ for $I_1$,
$\Omega_1 \in \Omega^{2,0}_{I_1}(X)$.
The nondegeneracy follows from the nondegeneracy for the $\omega_i$: namely, for any $v \in T^{1,0}_{I_1} X$,
\begin{equation}
  \Omega_1(v,\cdot) = 0 \implies \Omega_1(v+\bar{v}, \cdot) = 0 \implies \omega_2(v + \bar{v}, \cdot) = 0 \implies v + \bar{v} = 0 \implies v = 0.
\end{equation}
The remaining claims are obtained by cyclic permutations.
\end{pf}

\begin{cor}[\Hk manifolds have dimension $4n$] If $X$ is \hk then $\dim_\R X$ is a multiple of $4$.
\end{cor}
\begin{pf} There is a standard bit of ``symplectic linear algebra'' saying that a vector space with a
nondegenerate antisymmetric pairing is always 
even-dimensional (to prove it, one inductively constructs
a ``Darboux'' basis, in which the antisymmetric pairing
is a direct sum of blocks 
of the shape $\tiny \begin{pmatrix} 0 & 1 \\ -1 & 0 \end{pmatrix}$). 
Thus the existence of the holomorphic symplectic form $\Omega_1$ implies
that $T^{1,0}_{I_1} X$ has even complex dimension.
\end{pf}

\begin{prop}[Holomorphic symplectic form determines complex structure] \label{prop:hol-symp-to-complex}
Suppose $X$ is a manifold with $\dim_\R X = 2n$, 
with $\Omega \in \Omega^2_\C(X)$, such that $\de \Omega = 0$ and
\begin{equation}
 T_\C X = \ker \Omega \oplus \ker \overline\Omega. 
\end{equation}
Then there is a 
unique complex structure $I$ on $X$, for which $\Omega$
is a holomorphic symplectic form.
\end{prop}
\begin{pf} We define a complex-linear operator 
$I_\C$ on $T_\C X$ to act by $-\I$ on
$\ker \Omega$, and by $+\I$ on $\ker \overline\Omega$.
This $I_\C$ obeys $I_\C v = \overline{I_\C \bar{v}}$, so it is the 
complexification of a real-linear operator $I$ on $TX$,
which gives an almost complex structure.
The integrability of $I$ is equivalent to requiring
that $T^{0,1} X = \ker \Omega$ is an integrable distribution,
i.e. that if $v,w$ are sections of $\ker \Omega$ then $[v,w]$
is also a section of $\ker \Omega$.
This follows from $\de \Omega = 0$ and the covariant formula
for $\de$:
\begin{equation}
  \de \Omega(v,w) = v (\Omega(w)) - w (\Omega(v)) + \Omega([v,w]).
\end{equation}
\end{pf}

\begin{prop}[Explicit formula for the complex structures on a 
\hk manifold in terms of $\omega_i$]
If $X$ is \hk then
\begin{equation}
I_1 = \omega_3^{-1} \omega_2  
\end{equation}
and cyclic permutations.
(What this formula really means: view $\omega_2$ as a map $TX \to T^* X$
namely $v \mapsto \omega_2(v,\cdot)$,
and $\omega_3^{-1}$ as a map $T^* X \to TX$
namely $\omega_3(v,\cdot) \mapsto v$; then the composition
$\omega_3^{-1} \omega_2: TX \to TX$ is $I_1$.)
\end{prop}
\begin{pf}
What we have to check is that $\omega_3(I_1 v,\cdot) = \omega_2(v,\cdot)$.
But 
\begin{equation}
 \omega_3(I_1 v,\cdot) = g(I_3 I_1 v,\cdot) = g(I_2 v, \cdot) = \omega_2(v,\cdot) 
\end{equation}
as desired.
\end{pf}

\begin{cor}[The $\omega_i$ determine the \hk metric]
If $X$ is \hk then
\begin{equation}
g = -\omega_1 \omega_3^{-1} \omega_2  
\end{equation}
and cyclic permutations.
(Here similarly we view $g$ as a map $TX \to T^*X$,
namely $v \mapsto g(v,\cdot)$.)
\end{cor}

% \begin{prop}[Condition for forms $\omega_i$ to give a \hk metric] \label{prop:omega-hk-condition}
% Suppose $X$ is a smooth manifold with symplectic forms 
% $\omega_1, \omega_2, \omega_3$, obeying the condition that
% \begin{equation}
% -\omega_1 \omega_3^{-1} \omega_2 = -\omega_2 \omega_1^{-1} \omega_3 = -\omega_3 \omega_2^{-1} \omega_1,
% \end{equation}
% and that this quantity, $g$, 
% is \ti{positive definite} as a symmetric bilinear form.
% Then $g$ is a \hk metric on $X$,
% with $\omega_i$ the associated \kahler forms.
% \end{prop}

\begin{exercise}
Suppose $(X,g,I_1,I_2,I_3)$ is a \hk manifold. Fix any $\vec s = (s_1,s_2,s_3) \in S^2 \subset \R^3$, and set
\begin{equation}
  I_{\vec s} = \sum_{i=1}^3 s_i I_i, \qquad \omega_{\vec s} = \sum_{i=1}^3 s_i \omega_i.
\end{equation}
Show that $(X,I_{\vec s},g)$ is a \kahler manifold,
with \kahler form $\omega_{\vec s}$.
\end{exercise}
In other words, a \hk metric is \kahler for a whole 
$S^2$ of complex structures, not only three of them.
We can think of this $S^2$ as the
set of norm-$1$ imaginary quaternions.
Specifying $I_1$, $I_2$, $I_3$ is equivalent to 
specifying the whole collection of $I_{\vec s}$.

Note that the antipodal map acts in a simple way:
$I_{-\vec s} = - I_{\vec s}$, the opposite
complex structure of $I_{\vec s}$ --- i.e. 
the antipodal map exchanges holomorphic and antiholomorphic.

\begin{exercise}
Given a Riemannian manifold $(X,g)$ and 
a \hk structure thereon, specified by complex structures
$I_{\vec s}$, show that we get another
\hk structure by choosing an element $T \in SO(3)$
and defining 
\begin{equation}
I'_{\vec s} = I_{T \vec s}.
\end{equation}
Thus $SO(3)$ naturally acts on the set of
\hk manifolds.
\end{exercise}


\currentposition


\subsection{First examples}

\begin{example}[Flat quaternionic space] \label{exa:R4} The quaternions
$\bbH$ can be identified with $\R^4$
via the map
\begin{equation}
  x_0 + x_1 \I + x_2 \J + x_3 \K \mapsto (x_0,x_1,x_2,x_3).
\end{equation}
This makes $\bbH$ into a manifold of real 
dimension $4$.
The standard metric on $\R^4$ induces a metric
$g$ on $\bbH$.
If we identify $T_p \bbH \simeq \bbH$ 
in the obvious way,
the operations of left-multiplication by $\I$, $\J$ and $\K$
give complex structures $I_1$, $I_2$, $I_3$ on $\bbH$,
obeying the quaternion algebra.
Evidently these are all covariantly constant, so
$g$ is \kahler for all three of these complex
structures, and thus $\bbH$ is \hk.

The symplectic forms are
\begin{align}
  \omega_1 &= \de x_0 \wedge \de x_1 + \de x_2 \wedge \de x_3, \\
  \omega_2 &= \de x_0 \wedge \de x_2 + \de x_3 \wedge \de x_1, \\
  \omega_3 &= \de x_0 \wedge \de x_3 + \de x_1 \wedge \de x_2,
\end{align}
or more uniformly
\begin{equation} \label{eq:symplectic-forms-R4}
  \omega_i = \de x_0 \wedge \de x_i + \star \de x_i.
\end{equation}
The holomorphic symplectic form is
\begin{equation}
  \Omega_1 = \omega_2 + \I \omega_3 = \de w_1 \wedge \de z_1, \qquad w_1 = x_0 + \I x_1, \qquad z_1 = x_2 + \I x_3
\end{equation}
$w_1$ and $z_1$ are complex coordinates with respect to $I_1$.
Thus, in structure $I_1$, $\bbH$ is biholomorphic to $\C^2$.
\end{example}

\begin{exercise} \label{exc:flat-space-computing}
Verify the explicit formulas for the symplectic forms
$\omega_i$ on $\bbH$, and write a formula for $\omega_{\vec s}$.
\end{exercise}

$\bbH$ has a lot of symmetry. For example, $\bbH$ acts 
on itself by translations preserving the \hk structure.
Also the group $O(4)$ acts on $\bbH$ by isometries, but 
these do \ti{not} generally preserve the \hk structure.
However, we do have the following. The 
unit sphere in ${\mathbb H}$ is a Lie group,
which happens to be isomorphic to $SU(2)$.\footnote{The isomorphism
can be given explicitly by the formula
\begin{equation}
  x_0 + x_1 \I + x_2 \J + x_3 \K \mapsto \begin{pmatrix} x_0+x_1\I & x_2+x_3\I \\ -x_2+x_3\I & x_0-x_1\I \end{pmatrix}
\end{equation}
but we will not need to use this anywhere.}
Thus we have an action of $SU(2) \times SU(2)$ 
on ${\mathbb H}$ by
\begin{equation}
  (q, q') \cdot x = q x q'^{-1}.
\end{equation}
This gives a map $SU(2) \times SU(2) \to SO(4)$.
Said otherwise, $O(4)$ has two canonical $SU(2)$ subgroups,
which we call $SU(2)_L$ and $SU(2)_R$ (for ``left'' and ``right.'')
(Incidentally, this map has kernel $\{(1,1),(-1,-1)\} \simeq \Z_2$, 
thus gives an isomorphism $SO(4) \simeq (SU(2) \times SU(2)) / \Z_2$.)

\begin{exercise} \label{exc:su2r-action}
Show that $SU(2)_R$ acts on $\bbH$ by \ti{triholomorphic} 
isometries, i.e. isometries which are holomorphic for
all of $I$, $J$, and $K$.
\end{exercise}

\begin{exercise} \label{exc:su2l-action}
Show that the action of $T \in SU(2)_L$ on $\bbH$ has
\begin{equation}
  T^* I_{\vec s} = I_{T \vec s}
\end{equation}
On the right side, by $T \vec s$ we mean the \ti{conjugation}
action of the unit quaternion $T$ on the sphere of norm-$1$
imaginary quaternions (which gives the standard double-covering
$SU(2) \to SO(3)$.)
\end{exercise}

\begin{exercise}
Show that in any complex structure
$I_{\vec s}$, $\bbH$ is biholomorphic
to $\C^2$.
\end{exercise}

\begin{example}[Quotients of $\bbH$] \label{exa:H-quotients}
It follows from \autoref{exc:su2r-action} that, if we choose a subgroup
$\Gamma \subset SU(2)_R$, the quotient
$\bbH / \Gamma$ is a \hk orbifold: in particular, it
carries a natural \hk structure 
on the locus where it is a manifold. For example,
if $\Gamma$ is a discrete subgroup, it acts
freely away from the origin, so
\begin{equation}
X_\Gamma^\circ = (\bbH \setminus \{0\}) / \Gamma  
\end{equation}
is a \hk manifold.
However, this \hk manifold is \ti{incomplete},
since the origin is at finite distance.
\end{example}

\begin{example}[$\R^3 \times S^1$] \label{exa:R3S1}
Since translations preserve the \hk structure, we can divide $\bbH$
out by $\Z$ acting by translations 
\begin{equation}
 x_0 \to x_0 + 2 \pi n 
\end{equation}
to get another \hk manifold,
\begin{equation}
 X = \bbH / \Z \simeq \R^3 \times S^1.
\end{equation}
In structure $I_1$ we have
\begin{equation}
  \Omega_1 = -\I \frac{\de \cX_1}{\cX_1} \wedge \de z_1, \qquad \cX_1 = \exp(\I(x_0 + \I x_1)), \qquad z_1 = x_2 + \I x_3 
\end{equation}
The functions $(z_1,\cX_1)$ make $(X,I_1)$ 
biholomorphic to $\C \times \C^\times$.
\end{example}

\begin{exercise}
Show that $SO(3)$ acts by isometries on $X = \bbH / \Z$,
with
\begin{equation}
  T^* I_{\vec s} = I_{T \vec s}.
\end{equation}
\end{exercise}

\begin{exercise}
Show that for any $\vec s$, $X = \bbH / \Z$ with complex structure
$I_{\vec s}$ is biholomorphic to $\C \times \C^\times$.
\end{exercise}

\begin{example}[Incomplete Gibbons-Hawking spaces] \label{exa:incomplete-gibbons-hawking}
Now we generalize from $\R^3 \times S^1$ to a more general
\hk space with $U(1)$ action, i.e. we consider a principal
$U(1)$ bundle.

Fix some open set $U \subset \R^3$ and
let $V: U \to \R_{>0}$ be a positive harmonic function.\footnote{Our
conventions for the Laplace operator on $\Omega^0(\R^3)$ are:
$\Delta f = \de \star \de f = \sum_i \partial_i^2 f \, \de \vol$.}
Then $\Delta V = \de \star \de V = 0$, so if we write
\begin{equation}
  F = -2 \pi \star \de V
\end{equation}
then we have $\de F = 0$. 
Fix a principal $U(1)$ bundle $X$ over $U$, carrying a connection
$\Theta$ whose curvature is $F$. 
(Such an $X$ exists if and only if $[F/2\pi] \in H^2(U,\Z)$.) 
Concretely, we represent $\Theta$ as a $1$-form on $X$. 
If we choose a local
trivialization of the $U(1)$-bundle $X$ over a patch
$U_\alpha \subset U$, with local 
fiber coordinate $x_0 \in \R / 2 \pi \Z$, then $\Theta$ 
is locally of the form
\begin{equation}
 \Theta = A + \de x_0 
\end{equation}
with $A \in \Omega^1(U_\alpha)$, $\de A = F$.
\newcommand{\tTheta}{\tilde\Theta}
For convenience write $\tTheta = \Theta / 2\pi$.

We introduce three symplectic forms on $X$,
generalizing \eqref{eq:symplectic-forms-R4}:
\begin{equation}
  \omega_i = \tTheta \wedge \de x_i + V \star \de x_i.
\end{equation}
To check that these are indeed closed,
\begin{equation}
  \de \omega_i = - \star \de V \wedge \de x_i + \de V \wedge \star \de x_i = 0
\end{equation}
(the last equality because all of these forms are pulled back
from $\R^3$, and on $\R^3$ we always have $\star \alpha \wedge \beta = \alpha \wedge \star \beta$ for 1-forms $\alpha, \beta$.)
Then define
\begin{align}
  \Omega_1 &= \omega_2 + \I \omega_3 \\
  &= \tTheta \wedge \de z_1 +  V \star \de(x_2 + \I x_3)\\
  &= \tTheta \wedge \de z_1 +  V (\de x_3 \wedge \de x_1 + \I \de x_1 \wedge \de x_2)\\
  &= \tTheta \wedge \de z_1 + \I V \de x_1 \wedge \de z_1 \\
  &= V \alpha_1 \wedge \de z_1
\end{align}
where we introduced
\begin{equation}
  z_1 = x_2 + \I x_3, \qquad 
  \alpha_1 = V^{-1} \tTheta + \I \de x_1.
\end{equation}
Then $\ker \Omega_1$ is spanned by $\hat\partial_2 + \I \hat\partial_3$
and $2 \pi V \partial_0 + \I \hat\partial_1$, where $\partial_0$ 
is the globally defined generator of the $U(1)$ action 
(shifting $x_0$),  and $\hat \partial_i$ means
the parallel lift of $\partial_i$ from $\R^3$ to $X$, i.e. the lift
obeying $\Theta \cdot \hat\partial_1 = 0$. Thus
by \autoref{prop:hol-symp-to-complex}, $\Omega_1$ 
determines a complex structure $I_1$ on $X$, which acts
by $+\I$ on $\de z_1$ and $\alpha_1$.
In this structure, $\de z_1$ is of type $(1,0)$, so
$z_1$ is a holomorphic map,
\begin{equation}
  z_1: X \to \C.
\end{equation}

Morally this map makes $X$ into something like 
a $\C^\times$-bundle over a patch of $\C$ (generalizing
the $\C\times\C^\times$ which we got in \autoref{exa:R3S1}).
It is not quite a $\C^\times$-bundle in general
(since $U$ was an arbitrary open subset of $\R^3$), 
but it does at least have 
a holomorphic vector field tangent to the fibers,
$V \partial_0 - \I \hat \partial_1$.

Similarly define complex structures $I_2$, $I_3$.
Just as for $\R^3 \times S^1$ these obey the quaternion
relation $I_1 I_2 = I_3$.

Now we compute $g$ from $I_1$ and $\omega_1$:
\begin{equation}
  \omega_1 = V \re \alpha_1 \wedge \im \alpha_1 + V \re \de z_1 \wedge \im \de z_1
\end{equation}
which gives
\begin{align}
  g &= V ((\re \alpha_1)^2 + (\im \alpha_1)^2) + V ((\re \de z_1)^2 + (\im \de z_1)^2) \\
  &= V (V^{-2} \tTheta^2 + \de x_1^2) + V (\de x_2^2 + \de x_3^2) \\
  &= V \de \vec{x}^2 + V^{-1} \tTheta^2
\end{align}
% Note that the explicit index $1$ has disappeared from this formula;
% if we had computed $g$ from $I_2$ and $\omega_2$, or
% from $I_3$ and $\omega_3$, we would get the same result.
Thus the metric on $X$ given by
\begin{equation}
  g = V \de \vec{x}^2 + V^{-1} \tTheta^2
\end{equation}
is \hk, with \kahler forms $\omega_i$.

The principal $U(1)$ action on $X$ is by isometries preserving
the \hk structure (this is clear since nothing in $\omega_i$
depends on the fiber coordinates).
\end{example}

\begin{example}[Gibbons-Hawking spaces] \label{exa:gibbons-hawking}
Extending \autoref{exa:incomplete-gibbons-hawking}
we can allow $V$ to have quantized (distributional) sources
on $U$:
\begin{equation}
  \Delta V = - \sum_{i=1}^n \delta(\vec x - \vec x_i)
\end{equation}
In this case, consider a small $S^2$ around $\vec{x}_i$:
\begin{equation}
  \int_{S^2} \frac{F}{2\pi} = -\int_{S^2} \star \de V = -\int_{B^3} \de \star \de V = \int_{B^3} \delta(\vec x - \vec x_i) = 1
\end{equation}
so the $U(1)$ bundle $X$ restricted to this $S^2$ has 
degree $1$. $X$ thus doesn't extend as a $U(1)$ bundle
over the point $\vec x_i$.
Nevertheless it \ti{does} extend as a \hk manifold.
Indeed, near $\vec x_i$ we have
\begin{equation}
V = \frac{1}{4\pi\norm{\vec x - \vec x_i}} + \text{regular},
\end{equation}
so the circle fibers of $X$ are shrinking to zero
length, and it is possible to add
a single point over $\vec x_i$, in such a way 
that the total space is a \hk manifold with non-free 
$U(1)$ action, and the quotient is $U$.
This follows from the next (hard) exercise.
\end{example}

\begin{exercise} \label{exc:gibbons-hawking-R4}
Consider \autoref{exa:incomplete-gibbons-hawking} with
$U = \R^3 \setminus \{\vec 0\}$ and
$V(\vec x) = \frac{1}{4\pi\norm{\vec x}}$.
Show that $X$ with the metric $g$ is isometric
to $\R^4$ with one point deleted,
and that the \hk structure on $X$ extends
to a \hk structure on the full $\R^4$.
\end{exercise}

% This was also true of $\bbH$ of course 
% (\autoref{exc:flat-space-computing}); nevertheless, Taub-NUT space
% and $\bbH$ are definitely not the same \hk manifold!
% Thus there is more information in the \hk structure than
% in the set of complex manifolds one obtains by specializing.

\begin{example}[Eguchi-Hanson space]
This is a case of \autoref{exa:gibbons-hawking}
with two singularities. Fix two points
$\vec x_1, \vec x_2 \in \R^3$.
Take $U = \R^3$ and 
\begin{equation}
  V(\vec x) = \frac{1}{4\pi \norm{\vec x-\vec x_1}} + \frac{1}{4\pi \norm{\vec x-\vec x_2}}.
\end{equation}
Then $X$ is fibered over $\R^3$ with two degenerate fibers.
Let $\pi: X \to \R^3$ be the projection.
Then $\pi^{-1}(\vec x_1 \vec x_2)$ has the topology of $S^2$.
\end{example}

\insfig{higgs-bundles-3}{0.8}

\begin{exercise}
Show that this $S^2$ is a complex submanifold of $X$, with
respect to the two complex structures $I_{\vec s}$ where $\vec s$
is the direction from $\vec{x}_1$ to $\vec{x}_2$ or vice versa.
\end{exercise}

\begin{exercise}
Show that, in either of the complex structures 
$I_{\vec s}$ of the previous
exercise, $X$ is biholomorphic to $T^* \C\PP^1$.
\end{exercise}

Eguchi-Hanson space is our first example where the $I_{\vec s}$ do not all give rise to the same complex manifold:

\begin{exercise}
Show that, in any \ti{other} complex structure $I_{\vec s}$,
$X$ is \ti{not} biholomorphic to $T^* \C\PP^1$.
\end{exercise}

\begin{example}[Multi-Eguchi-Hanson spaces] \label{exa:multi-eguchi-hanson} \cite{MR520463} More generally we can fix a collection
of distinct points $\vec x_1, \dots, \vec x_k \in \R^3$ and take
\begin{equation}
  V(\vec x) = \sum_{i=1}^k \frac{1}{4 \pi \norm{\vec x - \vec x_i}}
\end{equation}
Then a straight line segment in $\R^3$ connecting two $\vec x_i$ (and not meeting any others)
gives an $S^2$ in $X$. This $S^2$ is a complex submanifold with respect to two of the complex structures $I_{\vec s}$, just as before.

Asymptotically, the metric on this space is approximately
what we would get by taking all $\vec x_i = \vec 0$. In that case
we would have simply 
\begin{equation}
 V(\vec x) = \frac{k}{4 \pi \norm{\vec x}}.
\end{equation}
After an overall rescaling by a factor of $k$, the metric
$g$ in this case is identical to the one in 
\autoref{exc:gibbons-hawking-R4}, \ti{except}
that the circle fibers are shorter by a factor of $k$.
In other words, asymptotically $g$ is approaching the metric of 
a quotient $\R^4 / \Z_k$. So, when all the $\vec x_i$ are distinct,
$g$ is a kind of \hk desingularization of $\R^4 / \Z_k$.
(The $\Z_k$ action here is the same one we saw 
in \autoref{exa:H-quotients}.)

A nice special case occurs when all of the $\vec x_i$ are collinear:
then we have a single complex structure in which $X$ contains $k-1$
holomorphic spheres $C_i$, with intersection numbers
$C_i \cdot C_{i+1} = 1$. In this complex structure $X$ is 
the \ti{minimal resolution} of the singularity $\C^2 / \Z_k$
(sometimes called a ``du Val singularity of type $A_{k-1}$,'' e.g.
because the intersection numbers of the $C_i$ make up the
Cartan matrix of type $A_{k-1}$.)
\end{example}

\begin{example}[ALE spaces]
One can consider the minimal resolutions $X_\Gamma$
of the singularities at the origin in \autoref{exa:H-quotients}.
Then $X_\Gamma$ is an honest smooth manifold, 
carrying a natural family 
of complete \hk metrics \cite{MR90d:53055}.
These metrics asymptotically approach
the metric we started with on $X_\Gamma^\circ$
(induced from the flat metric on $\bbH$);
thus the $X_\Gamma$ are called ``ALE spaces'',
for ``asymptotically locally Euclidean.''
In the case $\Gamma = \Z_k$ these \hk metrics are
just those of \autoref{exa:multi-eguchi-hanson};
for other $\Gamma$ they are not of Gibbons-Hawking type.
\fixme{How close are they?}
\end{example}

\begin{example}[Taub-NUT space]
This is a case of \autoref{exa:gibbons-hawking}
with one singularity. 
Take $U = \R^3$ and
\begin{equation}
  V(\vec x) = 1 + \frac{1}{4\pi \norm{\vec x}}.
\end{equation}
\fixme{...}
\end{example}

\begin{exercise} Show that Taub-NUT space is
biholomorphic to $\C^2$, in any of its complex
structures.
\end{exercise}

\fixme{reduced holonomy}

\fixme{Ricci-flatness}

\fixme{twistor family}

\printbibliography

\end{document}