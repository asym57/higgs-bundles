\documentclass[12pt,letterpaper,reqno]{amsart}

% \usepackage{mathtools}
\usepackage{epsfig}
\usepackage{amsmath}
\usepackage{amssymb}
\usepackage{amsthm}
\usepackage{indentfirst}
\usepackage{xspace}
\usepackage{multirow}
\usepackage{hyperref}
\usepackage{xcolor}
\definecolor{darkred}{rgb}{0.5,0.15,0.15}
\hypersetup{colorlinks=true,urlcolor=darkred,linkcolor=darkred,citecolor=darkred}
\usepackage{verbatim}
\usepackage[letterpaper,margin=1in,headheight=15pt]{geometry}
\usepackage{mathpazo}
\usepackage{tikz-cd}
\usepackage{booktabs}
\usepackage{framed}
\usepackage{float}
\usepackage{thmtools}

\definecolor{shadecolor}{rgb}{0.85,0.85,0.85}
% Bibliography formatting
\usepackage[bibstyle=authoryear-comp,labeldate=false,defernumbers=true,maxnames=20,firstinits=true,uniquename=init,dashed=false,backend=biber,sorting=none]{biblatex}

\DeclareNameAlias{sortname}{first-last}

\DeclareFieldFormat{url}{\url{#1}}
\DeclareFieldFormat[article]{pages}{#1}
\DeclareFieldFormat[inproceedings]{pages}{\lowercase{pp.}#1}
\DeclareFieldFormat[incollection]{pages}{\lowercase{pp.}#1}
\DeclareFieldFormat[article]{volume}{\textbf{#1}}
\DeclareFieldFormat[article]{number}{(#1)}
\DeclareFieldFormat[article]{title}{\MakeCapital{#1}}
\DeclareFieldFormat[inproceedings]{title}{#1}
\DeclareFieldFormat{shorthandwidth}{#1}

% Don't use "In:" in bibliography. Omit urls from journal articles.
\DeclareBibliographyDriver{article}{%
  \usebibmacro{bibindex}%
  \usebibmacro{begentry}%
  \usebibmacro{author/editor}%
  \setunit{\labelnamepunct}\newblock
  \MakeSentenceCase{\usebibmacro{title}}%
  \newunit
  \printlist{language}%
  \newunit\newblock
  \usebibmacro{byauthor}%
  \newunit\newblock
  \usebibmacro{byeditor+others}%
  \newunit\newblock
  \printfield{version}%
  \newunit\newblock
%  \usebibmacro{in:}%
  \usebibmacro{journal+issuetitle}%
  \newunit\newblock
  \printfield{note}%
  \setunit{\bibpagespunct}%
  \printfield{pages}
  \newunit\newblock
  \usebibmacro{eprint}
  \newunit\newblock
  \printfield{addendum}%
  \newunit\newblock
  \usebibmacro{pageref}%
  \usebibmacro{finentry}}

% Remove dot between volume and number in journal articles.
\renewbibmacro*{journal+issuetitle}{%
  \usebibmacro{journal}%
  \setunit*{\addspace}%
  \iffieldundef{series}
    {}
    {\newunit
     \printfield{series}%
     \setunit{\addspace}}%
  \printfield{volume}%
%  \setunit*{\adddot}%
  \printfield{number}%
  \setunit{\addcomma\space}%
  \printfield{eid}%
  \setunit{\addspace}%
  \usebibmacro{issue+date}%
  \newunit\newblock
  \usebibmacro{issue}%
  \newunit}


% Bibliography categories
\def\makebibcategory#1#2{\DeclareBibliographyCategory{#1}\defbibheading{#1}{\section*{#2}}}
\makebibcategory{books}{Books}
\makebibcategory{papers}{Refereed research papers}
\makebibcategory{chapters}{Book chapters}
\makebibcategory{conferences}{Papers in conference proceedings}
\makebibcategory{techreports}{Unpublished working papers}
\makebibcategory{bookreviews}{Book reviews}
\makebibcategory{editorials}{Editorials}
\makebibcategory{phd}{PhD thesis}
\makebibcategory{subpapers}{Submitted papers}
\makebibcategory{curpapers}{Current projects}

\setlength{\bibitemsep}{2.65pt}
\setlength{\bibhang}{.8cm}
\renewcommand{\bibfont}{\small}

\renewcommand*{\bibitem}{\addtocounter{papers}{1}\item \mbox{}\hskip-0.85cm\hbox to 0.85cm{\hfill\arabic{papers}.~~}}
\defbibenvironment{bibliography}
{\list{}
  {\setlength{\leftmargin}{\bibhang}%
   \setlength{\itemsep}{\bibitemsep}%
   \setlength{\parsep}{\bibparsep}}}
{\endlist}
{\bibitem}

\newenvironment{publications}{\section{\LARGE Publications}\label{papersstart}\vspace*{0.2cm}\small
\titlespacing{\section}{0pt}{1.5ex}{1ex}\itemsep=0.00cm
}{\label{papersend}\addtocounter{sumpapers}{-1}\refstepcounter{sumpapers}\label{sumpapers}}

\def\printbib#1{\printbibliography[category=#1,heading=#1]\lastref{sumpapers}}

% Counters for keeping track of papers
\newcounter{papers}\setcounter{papers}{0}
\newcounter{sumpapers}\setcounter{sumpapers}{0}
\def\lastref#1{\addtocounter{#1}{\value{papers}}\setcounter{papers}{0}}

% theorem environments
\declaretheoremstyle[bodyfont=\normalfont]{noital}

\declaretheorem[name=Theorem,numberwithin=section]{thm}
\declaretheorem[name=Definition,sibling=thm]{defn}
\declaretheorem[name=Proposition,sibling=thm]{prop}
\declaretheorem[name=Example,sibling=thm,style=noital]{example}
\declaretheorem[name=Exercise,numberwithin=section,style=noital,shaded={bgcolor={gray}{0.9}}]{exercise}

\numberwithin{equation}{section}

% macros for convenience
\newcommand{\tops}{\texorpdfstring}

\newcommand{\nid}{\noindent}

\newcommand{\fa}{{\mathfrak a}}
\newcommand{\fp}{{\mathfrak p}}
\newcommand{\fk}{{\mathfrak k}}
\newcommand{\fg}{{\mathfrak g}}
\newcommand{\fh}{{\mathfrak h}}
\newcommand{\fn}{{\mathfrak n}}
\newcommand{\fq}{{\mathfrak q}}
\newcommand{\fm}{{\mathfrak m}}
\newcommand{\fr}{{\mathfrak r}}

\newcommand{\cC}{\ensuremath{\mathcal C}}
\newcommand{\cG}{\ensuremath{\mathcal G}}
\newcommand{\cB}{\ensuremath{\mathcal B}}
\newcommand{\cL}{\ensuremath{\mathcal L}}
\newcommand{\cS}{\ensuremath{\mathcal S}}
\newcommand{\cF}{\ensuremath{\mathcal F}}
\newcommand{\cK}{\ensuremath{\mathcal K}}
\newcommand{\cZ}{\ensuremath{\mathcal Z}}
\newcommand{\cM}{\ensuremath{\mathcal M}}
\newcommand{\cO}{\ensuremath{\mathcal O}}
\newcommand{\cH}{\ensuremath{\mathcal H}}
\newcommand{\cX}{\ensuremath{\mathcal X}}
\newcommand{\cY}{\ensuremath{\mathcal Y}}
\newcommand{\cA}{\ensuremath{\mathcal A}}
\newcommand{\cI}{\ensuremath{\mathcal I}}

\newcommand{\R}{\ensuremath{\mathbb R}}
\newcommand{\C}{\ensuremath{\mathbb C}}
\newcommand{\PP}{\ensuremath{\mathbb P}}
\newcommand{\Z}{\ensuremath{\mathbb Z}}
\newcommand{\Q}{\ensuremath{\mathbb Q}}
\newcommand{\A}{\ensuremath{\mathbb A}}
\newcommand{\bbH}{\ensuremath{\mathbb H}}
\newcommand{\bbI}{\ensuremath{\mathbb I}}
\newcommand{\bS}{\ensuremath{\mathbb S}}

\newcommand{\half}{\ensuremath{\frac{1}{2}}}
\newcommand{\qtr}{\ensuremath{\frac{1}{4}}}
\newcommand{\bq}{{\mathbf q}}
\newcommand{\N}{{\mathcal N}}
\newcommand{\F}{{\mathcal F}}
\newcommand{\HH}{{\mathcal H}}
\newcommand{\LL}{{\mathcal L}}
\newcommand{\RR}{{\mathcal R}}
\newcommand{\V}{{\mathcal V}}
\newcommand{\dirac}{\!\!\not\!\partial}
\newcommand{\Dirac}{\!\!\not\!\!D}
\newcommand{\cE}{{\mathcal E}}
\newcommand{\vs}{\not\!v}
\newcommand{\kahler}{K\"ahler\xspace}
\newcommand{\hk}{hyperk\"ahler\xspace}
\newcommand{\del}{\ensuremath{\partial}}%I added (LF)
\newcommand{\delbar}{\ensuremath{\overline{\partial}}}%I added (LF)
\newcommand{\zbar}{\ensuremath{\overline{z}}}%I added (LF)
\newcommand{\rme}{\ensuremath{\mathrm{e}}}%I added (LF)

\newcommand{\breg}{{\cB_{\mathrm{reg}}}}
\newcommand{\bsing}{{\cB_{\mathrm{sing}}}}

\newcommand{\Top}{{\mathrm Top}}
\newcommand{\I}{{\mathrm i}}
\newcommand{\J}{{\mathrm j}}
\newcommand{\K}{{\mathrm k}}
\newcommand{\e}{{\mathrm e}}
\newcommand{\de}{\mathrm{d}}
\newcommand{\ab}{\mathrm{ab}}

\newcommand{\abs}[1]{\lvert#1\rvert}
\newcommand{\norm}[1]{\lVert#1\rVert}
\newcommand{\IP}[1]{\langle#1\rangle}
\newcommand{\dwrt}[1]{\frac{\partial}{\partial#1}}
\newcommand{\eps}{\epsilon}

\newcommand{\pa}{{\partial}}
\newcommand{\ti}[1]{\textit{#1}}

\newcommand{\fro}{\overline{\underline{\Omega}}}

\newcommand{\unif}{{\mathrm {unif}}}
\newcommand{\fsl}{\mathfrak{sl}}

\newcommand{\hb}{harmonic bundle}

\DeclareMathOperator{\ad}{ad}
\DeclareMathOperator{\im}{Im}
\DeclareMathOperator{\re}{Re}
\DeclareMathOperator{\Tr}{Tr}
\DeclareMathOperator{\End}{End}
\DeclareMathOperator{\Hom}{Hom}
\DeclareMathOperator{\Aut}{Aut}
\DeclareMathOperator{\Sym}{Sym}
\DeclareMathOperator{\diag}{diag}

\newcommand{\fixme}[1]{{\color{blue}{[#1]}}}


% \mathtoolsset{showonlyrefs}

\bibliography{higgs-bundles}

\begin{document}

\title{Moduli of Higgs bundles}
\date{}

\maketitle

\setcounter{page}{1}

\section{Syllabus}

Moduli spaces of Higgs bundles are extremely 
geometrically rich.
The aim of the course is to learn as much as we can about 
the topology and geometry of these spaces. One 
idiosyncratic point is that we will focus more than 
usual on the fact that these spaces are \ti{\hk} 
(as opposed to merely holomorphic symplectic),
and our point of view will be more differential-geometric
than algebraic-geometric.

A very rough plan (probably too much to really cover,
so some of the things with question marks will have 
to be dropped; also some things will probably 
have to be done in a different order):
\begin{itemize}
  \item speedy review of basic definitions in complex and \kahler geometry
  \item basic notions of \hk geometry, hyperholomorphic bundles, twistor spaces (Hitchin's theorem)
  \item examples of \hk spaces:
  \begin{itemize}
  \item \hk and twistor geometry of $\R^4$, $\R^3 \times S^1$, $\R^2 \times T^2$, $T^* \C\PP^1$
  \item Gibbons-Hawking spaces?
  \item ALE spaces?
  \item Ooguri-Vafa manifold?
  \item Atiyah-Hitchin manifold?
  \item complex coadjoint orbits?
  \item moduli space of instantons?
  \item moduli space of solutions of Nahm equations?
  \item cotangent bundles of Riemann surfaces (incomplete)?
  \item hypertoric varieties?
  \item K3 surface?
  \end{itemize}
  \item symplectic and \hk quotients
  \item moduli space of Higgs bundles for the group $G = U(1)$ (incl. speedy review of abelian Hodge theory)
  \item moduli spaces of semistable $G$-Higgs bundles
  \item Hitchin's integrable system, spectral curves
  \item SYZ mirror symmetry?
  \item nonabelian Hodge theorem
  \item \hk structure on moduli of Higgs bundles
  \item Morse function, computation of Betti numbers
  \item asymptotic construction of harmonic bundles and \hk metric
  \item bundles with parabolic structure
  \item abelianization, cluster coordinates?
  \item exact (conjectural) description of \hk metric?
  \item $P=W$ conjecture?
  \item role in geometric Langlands program?
  \item role in $\N=2$ supersymmetric QFT?
\end{itemize}


\section{Complex and \kahler geometry: a quick review}

There are many references for this material: one
good one is \cite{MR2093043}.


\subsection{Complex manifolds} 
In this section $X$ is a smooth manifold.

\begin{defn}
An \ti{almost complex structure} on $X$ is a smooth section $I$ of $\End(TX)$ with $I^2 = -1$. An \ti{almost complex manifold} is a pair $(X,I)$ where $I$ is an almost complex structure.
\end{defn}

\begin{example} $\C^n$ has a canonical almost complex structure $I$, as follows. Each tangent space $T_p \C^n \simeq \R^{2n}$ canonically; $I$ is multiplication by $\I$.
Writing $z_i = x_i + \I y_i$, and taking the basis
$\{\partial_{x_1}, \partial_{x_2}, \dots, \partial_{x_n}, \partial_{y_1}, \partial_{y_2}, \dots, \partial_{y_n}\}$ for for $T_p \C^n$, we have \fixme{check sign}
\begin{equation}
   I = \begin{pmatrix} {\bf 0}_{n \times n} & {\bf 1}_{n \times n} \\ -{\bf 1}_{n \times n} & {\bf 0}_{n \times n} \end{pmatrix}.
\end{equation}
\end{example}

If $X$ has real dimension $2n$, an almost complex
structure $I$ equips $TX$ with the structure of
\ti{complex} vector bundle over $X$, of rank $n$,
and we say the \ti{complex dimension} $\dim_\C X$ is
$n$.

 \begin{defn} If $(X,I_X)$ and $(Y,I_Y)$ are almost
 complex manifolds, a \ti{holomorphic map} $\phi: X \to Y$
is one obeying
\begin{equation}
 I_Y \circ \de \phi = \de \phi \circ I_X.  
\end{equation}
\end{defn}

\begin{defn} An almost complex structure $I$ on $X$ is \ti{integrable} if there is a covering of $X$ by open sets $U_\alpha$ with holomorphic maps $\phi_\alpha: U_\alpha \to \C^n$. A \ti{complex manifold} is an almost complex manifold $(X,I)$ with $I$ integrable.
\end{defn}

\begin{exercise} Show that this is equivalent to the usual definition of a complex manifold as a space
$X$ with a covering by charts $\phi_\alpha: U_\alpha \to \C^n$, where the transition maps are holomorphic (obey Cauchy-Riemann equations).
\end{exercise}

There are various equivalent ways of formulating the integrability condition. One which will be useful for us is:
\begin{prop} $I$ is integrable if and only if the \ti{Nijenhuis tensor} $N_I \in C^\infty(\wedge^2 T^*X \otimes TX)$,
\begin{equation}
N_I(v,w) = [v,w] + I[v,Iw] + I[Iv,w] - [Iv,Iw],
\end{equation}
vanishes:
\begin{equation}
N_I = 0.  
\end{equation}
\end{prop}

\subsection{Type decompositions}
Suppose $(X,I)$ is a complex manifold.
We have a decomposition
of $T_\C X = TX \otimes_\R \C$,
\begin{equation}
T_\C X = T^{1,0} X \oplus T^{0,1} X
  \end{equation}
where $T^{1,0} X$ and $T^{0,1} X$ are 
respectively the
$+\I$ and $-\I$ eigenspaces of $I$.
Both $TX$ and $T^{1,0}X$ are complex vector bundles 
of rank $n$; it is sometimes convenient to identify
them, by projection on the $(1,0)$ part.

\begin{exercise}
Show that this is indeed an isomorphism 
of complex vector bundles.
\end{exercise}

There is also a dual decomposition
\begin{equation}
T^*_\C X = (T^*)^{1,0} X \oplus (T^*)^{0,1} X,
  \end{equation}
which induces
\begin{equation}
\wedge^* T^*_\C X = \bigoplus_{p,q=0}^n \wedge^{p,q} T^* X , \qquad \Omega^*_\C X = \bigoplus_{p,q=0}^n \Omega^{p,q}(X)
\end{equation}
and a corresponding decomposition
\begin{equation}
\de = \partial + \bar\partial, \quad \partial: \Omega^{p,q}(X) \to \Omega^{p+1,q}(X), \quad \bar\partial: \Omega^{p,q}(X) \to \Omega^{p,q+1}(X).
\end{equation}

\subsection{Holomorphic vector bundles}

In this section $(X,I)$ is always a complex manifold.

\begin{defn} A \ti{holomorphic vector bundle} over $X$ is a complex vector bundle $E$ over $X$, equipped with an operator
\begin{equation}
  \bar\partial_E :  \Omega^{p,q}(E) \to \Omega^{p,q+1}(E)
\end{equation}
obeying
\begin{equation}
  \bar\partial_E (\alpha \psi) = (\bar\partial_E \alpha) \psi + (-1)^{\abs{\alpha}} \alpha \wedge \bar\partial_E \psi
\end{equation}
and the integrability condition
\begin{equation}
  \bar\partial_E^2 = 0.
\end{equation}
\end{defn}

\begin{exercise} Show that a structure of holomorphic vector bundle on $E$ is equivalent to a maximal atlas 
of preferred trivializations of $E$, such that
the transition maps 
$U_\alpha \cap U_\beta \to GL(r,\C)$ 
are holomorphic.
\end{exercise}

% \begin{example} The trivial bundle $X \times \C^r$
% carries a canonical holomorphic structure.
% \end{example}

\begin{example} The tangent bundle $TX$ carries
a canonical structure of holomorphic vector bundle.
(Indeed, the coordinate charts give rise to 
preferred trivializations corresponding 
to the bases $\{\partial_{z_1}, \dots, \partial_{z_n} \}$
for $TX \simeq T^{1,0} X$, and the transition 
maps are given 
by the Jacobians, which are holomorphic.)
% It can be described as follows: for each coordinate
% chart $\phi_\alpha: U \to \C^n$, we get an isomorphism
% $\de \phi_\alpha: TU \to T \C^n$, and under this 
% isomorphism $\bar\partial_{TX}$ is identified
% with the canonical holomorphic structure.
\end{example}

\begin{defn}
If $E$ is a holomorphic vector bundle over $X$,
a connection $D$ in $E$ is \ti{compatible with the
holomorphic structure} if, for all 
$\psi \in \Omega^0(E)$, the $(0,1)$ part of $D \psi$ 
is $\bar\partial_E \psi$.
\end{defn}

\begin{defn}
If $E$ is a holomorphic vector bundle over $X$ with
a Hermitian metric $h$, the \ti{Chern connection} 
in $E$ is the unique connection which is 
$h$-unitary and compatible with the holomorphic
structure.
\end{defn}

\subsection{Hermitian and \kahler metrics}

In this section $(X,I)$ is always a complex manifold.

\begin{defn} A Hermitian metric on $X$ is a Riemannian
metric $g$ obeying
$$ g(v,w) = g(Iv,Iw). $$
\end{defn}
Equivalently, with respect to the decomposition 
\begin{equation}
  \Sym^2 (T_\C X) = \Sym^{2,0} TX \oplus \Sym^{1,1} TX \oplus \Sym^{0,2} TX,
\end{equation}
we have $g \in \Sym^{1,1} TX$, i.e. $g$ is of ``type $(1,1)$.''.

\begin{defn} If $g$ is a Hermitian metric on $X$, 
the \ti{fundamental form} $\omega \in \Omega^2(X)$
is
\begin{equation}
  \omega(v,w) = g(Iv,w).
\end{equation}
\end{defn}

\begin{exercise} If $g$ is a Hermitian metric on $X$,
verify that
\begin{equation}
  h = g - \I \omega
\end{equation}
defines a Hermitian metric on the complex vector
bundle $TX$.
\end{exercise}

\begin{defn} If $g$ is a Hermitian metric on $X$,
$g$ is \kahler if the corresponding $\omega$ obeys
\begin{equation}
  \de \omega = 0.
\end{equation}
In this case we say $(X,I,g)$ is a \kahler manifold,
and $\omega$ is the \kahler form.
\end{defn}

The \kahler property has various useful consequences,
some local and some global. Here we recall some
of the local ones:
\begin{prop}
If $(X,I,g)$ is a \kahler manifold, with
\kahler form $\omega$, then:
\begin{itemize}
  \item $\omega$ is a symplectic form on $X$,
  \item $\frac{\omega^n}{n!}$ is the volume form induced by $g$,
  \item the Levi-Civita connection on $TX$ agrees with the Chern connection,
  \item $\omega$ and $I$ are covariantly constant for the Levi-Civita connection.
\end{itemize}

\end{prop}

Finally we quickly recall the notion of special holonomy. Recall that for any Riemannian metric $g$ the parallel transport of Levi-Civita preserves $g$, so that 
for any $p \in X$ the holonomy group $Hol_g(p) \subset GL(T_p X)$ 
is contained in the subgroup $O(h_p) \simeq O(2n)$.
For a \kahler metric, since the Chern connection agrees with the 
Levi-Civita connection, the parallel transport of Levi-Civita preserves the Hermitian metric
$h$ on the complex vector bundle $TX$. Thus, for any $p \in X$, the holonomy group
$Hol_g(p) \subset GL(T_p X)$ is contained in the smaller group $U(h_p) \simeq U(n)$.
This proves one-half of the following:

\begin{prop} Given any Riemannian metric $g$ on a manifold $M$ of dimension $2n$, $g$ is a \kahler metric 
(for some complex structure $I$ on $M$) if and only if
the holonomy group at a point is contained in
a subgroup isomorphic to $U(n)$.
\end{prop}


\section{Hyperk\"ahler manifolds}

\begin{defn} A \hk manifold is a Riemannian 
manifold $(X,g)$,
equipped with three complex structures $I$, $J$, $K$, obeying the algebra of the quaternions ($IJ = -JI = K$),
such that $g$ is Hermitian and \kahler with respect to
any of $I$, $J$, $K$.
\end{defn}

It is crucial that we require the \ti{single} metric
$g$ to be \kahler for \ti{all three} complex structures.
We denote the three corresponding
symplectic forms $\omega_I, \omega_J, \omega_K$.

\begin{example} The space
$\bbH$ can be identified with $\R^4$
via the map
\begin{equation}
  a + b \I + c \J + d \K \mapsto (a,b,c,d).
\end{equation}
This makes $\bbH$ into a manifold of real 
dimension $4$.
The standard metric on $\R^4$ induces a metric
$g$ on $\bbH$.
If we identify $T_p \bbH \simeq \bbH$ 
in the obvious way,
the operations of multiplication by $\I$, $\J$ and $\K$
give complex structures $I$, $J$, $K$ on $\bbH$,
obeying the quaternion algebra.
The metric $g$ is \kahler for all three of these complex
structures. Thus $\bbH$ is \hk, in a canonical way.
\end{example}

\begin{exercise}
Write explicit formulas for the symplectic forms
$\omega_I$, $\omega_J$, $\omega_K$ on $\bbH$.
\end{exercise}

The group $O(4)$ acts on $\bbH$ by isometries, but 
these do \ti{not} generally preserve the \hk structure.
However, we do have the following:
\begin{exercise}
Recall the identification of Lie groups, $SO(4) \simeq (SU(2) \times SU(2)) / \Z_2$. Thus $SO(4)$ has two canonical $SU(2)$ subgroups. Show that one of these
subgroups acts on $\bbH$ by \ti{triholomorphic} isometries, i.e. isometries which are holomorphic for
all of $I$, $J$, and $K$.
\end{exercise}

\begin{example}
It follows that, if we choose a subgroup
$\Gamma \subset SU(2)$, the quotient
$\bbH / \Gamma$ carries a natural \hk structure 
at least where it is a manifold. For example,
if $\Gamma$ is a discrete subgroup, it acts
freely away from the origin, so
$X_\Gamma^\circ = (\bbH \setminus \{0\}) / \Gamma$
is a \hk manifold.
However, this \hk manifold is \ti{incomplete},
since the origin is at finite distance.
We could include the origin and say we have 
a \hk \ti{orbifold}. But there is an alternative: 
one can consider the so-called 
\ti{minimal resolution} of the singularity at the 
origin; call this $X_\Gamma$;
then $X_\Gamma$ is an honest smooth manifold, 
carrying a natural family 
of \ti{complete} \hk metrics \cite{MR90d:53055}.
These metrics asymptotically approach
the metric we started with on $X_\Gamma^\circ$
(induced from the flat metric on $\bbH$);
thus the $X_\Gamma$ are called ``ALE spaces'',
for ``asymptotically locally Euclidean.''
\end{example}

\fixme{reduced holonomy}

\fixme{holomorphic symplectic form}

\fixme{Ricci-flatness}

\fixme{twistor family}

\printbibliography

\end{document}
